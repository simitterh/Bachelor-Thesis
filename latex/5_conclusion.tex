There exists an investment strategy that the investor would want to choose in order to maximize the total expected utility. Several observations are important to find this optimum:\\

\begin{itemize}
    \item Consumption will be strictly positive both today and tomorrow, due to the property of the utility function that the first unit of consumption has infinite marginal value. \\
    
    \item The utility function is also strictly monotonously increasing, therefore the capital and the return on investment are always completely exhausted at the optimum.\\
    
    \item Scenario 2, where the investor has a risky and a riskless investment option, can be divided into four cases: The portfolio is either built from both investment options or from only one of the two. It is also an option to consume the whole capital immediately and not invest anything, however, this can be ruled out for an optimal investment-consumption strategy.\\
    I derive simple conditions which include asset prices, the discount factor, state probabilities and state-contingent asset payoffs that determine uniquely which of the cases applies.\\
    
    \item The optimal investment-consumption strategy leads to equal marginal utilities for immediate consumption and for each investment option exerted, i.e., for an additional unit of capital there must be the same additional utility whether it is invested or immediately consumed.\\
    The marginal utility with respect to an unused investment option is either equal to or smaller than the other marginal utilities.\\
    
    \item If only one investment option is used, the problem to be solved is one-dimensional. The solution can thus be found as a function of the discount factor, the prices, tomorrow's payoffs and state probabilities simply by rearranging the equation for the investment-consumption ratio.\\
    If both investment options are used, the optimality condition consists of two equations, which are both two-dimensional. Numerical methods can be used to obtain a solution.\\
    
    \item With iso-elastic utility, the optimal investment-consumption ratio can be found as a function of the discount factor, the expected gross return and the parameter of relative risk aversion. It is independent of the available capital and is therefore independent of scale.\\
    An increase in gross return leads a highly risk averse agent to invest less and consume more immediately. A slightly risk averse agent would react the other way around.\\
    A myopic investor makes his investment decision based solely on the discount factor and independently of the gross return.\\
    
    \item The impact of a mean preserving spread on the optimal investment decision depends on relative risk aversion: \\
    Highly risk-averse investors fear low outcomes and try to compensate for them by investing more.\\
    Slightly risk averse investors would consume more immediately, since they have a limited utility minimum.\\
    
    \item For those investments, which are part of the optimal portfolio, the price of the investment at the optimum must be equal to the expected discounted value of the asset's payoff, where discounting is done with the individual stochastic discount factor of the investor. If an asset is not in the optimal portfolio, the stochastic discount factor helps to determine a lower bound to the asset's price. With incomplete markets, the stochastic discount factors of the individual investors do not have to coincide, but they still agree on the prices of the financial instruments.
\end{itemize}